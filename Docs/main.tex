\documentclass[twoside,11pt]{article}
\usepackage[spanish]{babel}
\usepackage{xcolor}
\usepackage{style}
\usepackage{tabularx}
\usepackage{csquotes}
\usepackage{subcaption}
\usepackage{wrapfig}
\usepackage{graphicx}
\usepackage{float}
\usepackage{apacite}

\newcommand{\myname}{Joaquín Badillo Granillo}

\jmlrheading{Noviembre 29}{2023}{Movilidad Urbana}{TC2008B}
\ShortHeadings{Movilidad Urbana}{}

\begin{document}
\renewcommand{\tablename}{Tabla}

\title{Simulación Multiagente de Movilidad Urbana}

\author{\name \myname \email \href{mailto:a01026364@tec.mx}{a01026364@tec.mx} \\
       \addr Escuela de Ingeniería y Ciencias\\
       Tecnológico de Monterrey\\
       Campus Santa Fe
       \AND
       \name Pablo Bolio Pradilla \email \href{mailto:a01782428@tec.mx}{a01782428@tec.mx} \\
       \addr Escuela de Ingeniería y Ciencias\\
       Tecnológico de Monterrey\\
       Campus Santa Fe
       }

\maketitle

\begin{abstract}
    
\end{abstract}

\begin{keywords}
  Simulación, Multiagentes, Agentes Reactivos, Movilidad
\end{keywords}

\section{Introducción}
% Simulación multiagente para tráfico
El rápido crecimiento urbano ha dado como resultado altas densidades poblacionales.
En zonas centralizadas como la Ciudad de México, aunque la cercanía de los servicios
es muy alta, la cantidad de vehículos que se trasladan por las calles ha hecho que
el tiempo para llegar a dichos servicios sea muy largo en comparación con su distancia.

\subsection{Propuesta de Solución}
En este proyecto se consideró el problema de la movilidad urbana a partir de una simulación
multiagente utilizando la librería de mesa \cite[]{python-mesa-2020}. En esta simulación,
los agentes toman decisiones con un cierto grado de autonomía a partir de un conjunto de reglas
finitas y sencillas, como el respetar las señales de tránsito (en particular los semáforos) y 
la navegación a partir de las rutas más cortas para alcanzar un destino, 

Es entonces de nuestro interés observar si a partir de estas sencillas reglas y limitando el entendimiento 
que tienen los agentes del ambiente a pequeñas vecindades, pueden emerger comportamientos egoístas que resulten
en congestionamientos. Asímismo, se planteó una simulación en la que es posible modificar la afluencia de agentes,
de tal forma que se puedan observar los límites que tiene el diseño de una ciudad tras un incremento en la densidad
poblacional.

\section{Diseño de Agentes}

\begin{enumerate}
    \item Esperar
    
    \item Moverse

    \item Calcular ruta

    \item Llegar a destino
\end{enumerate}


\subsection{Objetivo}
\subsection{Capacidad Efectora (Actuadores)}
\subsection{Percepción}
\subsection{Proactividad}
\subsection{Métricas de Desempeño}
\section{Arquitectura de Subsunción}\label{sec:arch}
\begin{table}[ht]
    \centering
    \begin{tabular}{c| >{\raggedright}p{0.3\textwidth} >{\raggedright}p{0.3\textwidth} >{\raggedright\arraybackslash}p{0.3\textwidth}|}
    \cline{2-4}
      & \textbf{Evento} & \textbf{Condiciones} & \textbf{Acción} \\\cline{2-4}
     4 & Nada & Nada & Calcular ruta \\\cline{2-4}
     3 & Seguir ruta & Celda disponible (es su destino)  & Llegar a destino\\\cline{2-4}
     2 & Seguir ruta & Celda disponible & Moverse siguiendo ruta\\\cline{2-4}
     1 & Seguir ruta & Celda ocupada por otro coche & Calcular ruta (actualizando costos para las celdas vecinas)\\\cline{2-4}
     0 & Seguir ruta & En un cruce con semáforo en estado rojo & Esperar \\\cline{2-4}
    \end{tabular}
    \caption{Arquitectura de subsunción para los coches}
    \label{tab:arch}
\end{table}

\section{Características del Ambiente}
\section{Conclusiones}

\newpage

\bibliographystyle{apacite}
\bibliography{refs}
\end{document}