\documentclass[twoside,11pt]{article}
\usepackage[spanish]{babel}
\usepackage{xcolor}
\usepackage{style}
\usepackage{tabularx}
\usepackage{csquotes}
\usepackage{subcaption}
\usepackage{wrapfig}
\usepackage{graphicx}
\usepackage{float}
\usepackage{apacite}
\usepackage{bm}
\usepackage{pgfplots}

\pgfplotsset{
  compat=newest,
  plot coordinates/math parser=false,
  every axis plot/.append style={line width=1.5pt}
}
\newlength\figureheight
\newlength\figurewidth


\newcommand{\myname}{Joaquín Badillo Granillo}

\jmlrheading{Noviembre 29}{2023}{Movilidad Urbana}{TC2008B}
\ShortHeadings{Movilidad Urbana}{}

\begin{document}
\renewcommand{\tablename}{Tabla}

\title{Tolerance is All You Need}

\author{\name \myname \email \href{mailto:a01026364@tec.mx}{a01026364@tec.mx} \\
       \addr Escuela de Ingeniería y Ciencias\\
       Tecnológico de Monterrey\\
       Campus Santa Fe
       \AND
       \name Pablo Bolio Pradilla \email \href{mailto:a01782428@tec.mx}{a01782428@tec.mx} \\
       \addr Escuela de Ingeniería y Ciencias\\
       Tecnológico de Monterrey\\
       Campus Santa Fe
       }

\maketitle

\vspace{-16pt}

\begin{center}
  \textbf{Bajo la instrucción de:}
  
  \vspace{8pt}

  \begin{tabular}{cc}
    \hspace{34pt}Octavio Navarro Hinojosa\hspace{34pt} & \hspace{34pt}Gilberto Echeverría Furió\hspace{34pt} \\
    \href{mailto:octavio.navarro@tec.mx}{octavio.navarro@tec.mx} & \href{mailto:g.echeverria@tec.mx}{g.echeverria@tec.mx} 
  \end{tabular}
\end{center}

  \vspace{14pt}

\section{Introducción}
El rápido crecimiento urbano ha dado como resultado altas densidades poblacionales.
En zonas centralizadas como la Ciudad de México, aunque la cercanía de los servicios
es muy alta, la cantidad de vehículos que se trasladan por las calles ha hecho que
el tiempo para llegar a dichos servicios sea muy largo en comparación con su distancia.

\subsection{Propuesta de Solución}
En este proyecto se consideró el problema de la movilidad urbana a partir de una simulación
multiagente utilizando la librería de mesa \cite[]{python-mesa-2020}. En esta simulación,
los agentes toman decisiones con un cierto grado de autonomía a partir de un conjunto de reglas
finitas y sencillas, como el respetar las señales de tránsito (en particular los semáforos) y 
la navegación a partir de las rutas más cortas para alcanzar un destino.

Es entonces de nuestro interés observar si a partir de estas sencillas reglas y limitando el entendimiento 
que tienen los agentes del ambiente a pequeñas vecindades, pueden emerger comportamientos egoístas que resulten
en congestionamientos. Asímismo, se planteó una simulación en la que es posible modificar la afluencia de agentes,
de tal forma que se puedan observar los límites que tiene el diseño de una ciudad tras un incremento en la densidad
poblacional.

\section{Diseño de Agentes}

Para que los agentes puedan interactuar con su entorno, se representaron calles, edificios, destinos, semáforos y vehículos
como agentes.

Tanto los semáforos como los vehículos son capaces de cambiar su estado, pero únicamente los segundos tienen
un objetivo: que es llegar su destino. Por esta razón, el diseño que es de nuestro interés es el de los vehículos, el cual
se describe a detalle a continuación.

\subsection{Capacidad Efectora (Actuadores)}

Las acciones que puede realizar un vehículo están limitadas a:

\begin{enumerate}
    \item Esperar.
    
    \item Moverse.

    \item Calcular una ruta.

    \item Llegar a destino.
\end{enumerate}

Esperar, moverse y llegar a su destino, son acciones sencillas y fáciles
de implementar; mientras que el calculo de rutas permite una mayor libertad
creativa. En esta simulación, los agentes calculan rutas a partir del algoritmo 
de A*, usando generalmente el cuadrado de la norma $\ell^{2}$ como función de costos y 
la norma $\ell^{1}$ como heurística. En caso de recalcular una ruta, un agente puede
además usar la información de sus vecinos más próximos para incrementar el costo
entre celdas. Cabe destacar que también se consideró el uso de la norma $\ell^{\infty}$
como función de costos, ya que los movimientos en el software de simulación asemejan
en gran medida un tablero de ajedrez y no se observó ningún cambio en los resultados
obtenidos por A*.

\subsection{Percepción}
Los vehículos son únicamente capaces de observar una vecindad en un radio de $\sqrt{2}$
unidades (usando la norma $\ell^{2}$ como métrica). No obstante cuentan con un conocimiento
previo de las calles que les permite calcular rutas eficientes (asumiendo que no hay otros
vehículos).

\subsection{Proactividad}
Contemplando el objetivo principal de la simulación, que es alcanzar el destino en el menor tiempo posible, 
teniendo en cuenta que no son omniscientes los agentes, es decir, conocen las rutas, pero les es imposible saber 
si hay congestión en la ruta que seguirá. Por lo que los conductores vuelven a calcular la ruta cuando no es capaz de moverse 
y dependiendo de la nueva ruta adquirida es el comportamiento que sigue. Previendo que en momentos es mejor hacer una ruta absurdamente 
larga que esperar a que se descongestione una calle, los agentes cuentan con paciencia, la cual se reduce cada paso en el que esperan. 
En el momento en el que la paciencia esté por debajo de cierto rango, los agentes aceptan cualquier ruta como óptima. 
Lo que permite que todos lleguen a su destino evitando causar cuellos de botella.

\subsection{Métricas de Desempeño}
Si fuera de nuestro interés determinar el desempeño individual de un agente, una métrica relevante sería 
la cantidad de episodios que le tomó llegar a su destino, aunque debido a que la distancia que tienen que 
recorrer varía es aún mejor evaluarlos individualmente como
\begin{equation}
  \frac{\mathrm{Episodios \ Transcurridos}}{\langle\mathrm{Origen}, \mathrm{Destino}\rangle_{\ell^{\infty}}},
\end{equation}

donde $\langle \bm{x}, \bm{y}\rangle_{\ell^{\infty}}$ representa la distancia de Chebyshev entre 2
puntos $\bm{x}$, $\bm{y}$.

No obstante, como es de nuestro interés determinar el rendimiento de los agentes a un nivel global, es más
razonable evaluar el rendimiento que tienen de forma colaborativa y como responden a cambios en el flujo de
entrada de agentes al sistema. Por esta razón se decidió que para un episodio en particular $t$, la métrica de desempeño $R$
sea el número de agentes totales que han llegado a su destino

\begin{equation}\label{eq:metric}
  R(t) = \mathrm{N\acute{u}mero \ de \ llegadas \ despu\acute{e}s \ de \ } t \ \mathrm{episodios}.
\end{equation}

Evidentemente si se tiene un mayor flujo de agentes y estos logran colaborar de forma efectiva, se espera que
más agentes lleguen a sus destinos en un menor número de episodios. También cabe destacar que la métrica de desempeño
es más interesante cuando la simulación llega a su condición de cierre, que es cuando se deben agregar agentes pero ninguna esquina
del mapa (que es donde aparecen) está disponible.

Para poder variar el flujo de entrada en la simulación, esta cantidad se representó como un parámetro de línea de comandos (o una variable de entorno) 
para el servidor, que determina el número de episodios que debe transcurrir antes de volver a agregar agentes. Asimismo es posible especificar
dicho valor desde la ruta {\tt{/init}} del API, pasando el parámetro {\tt{cycles}} desde un formulario web.

\section{Arquitectura de Subsunción}\label{sec:arch}

\begin{table}[ht]
    \centering
    \begin{tabular}{c| >{\raggedright}p{0.3\textwidth} >{\raggedright}p{0.3\textwidth} >{\raggedright\arraybackslash}p{0.3\textwidth}|}
    \cline{2-4}
      & \textbf{Evento} & \textbf{Condiciones} & \textbf{Acción} \\\cline{2-4}
     7 & Nada & Nada & Calcular ruta \\\cline{2-4}
     6 & Seguir ruta & Celda disponible & Moverse siguiendo ruta\\\cline{2-4}
     5 & Seguir ruta & Celda ocupada por otro coche, es una diagonal y la paciencia es menor o igual a 0 & Recalcular ruta (actualizando costos)\\\cline{2-4}
     4 & Seguir ruta & Celda ocupada por otro coche, es una diagonal y la paciencia es positiva & Esperar (restando paciencia)\\\cline{2-4}
     3 & Seguir ruta & La paciencia es negativa y llegó a un límite & Intentar moverse a una celda disponible aleatoria y recalcular ruta\\\cline{2-4}
     2 & Seguir ruta & Celda disponible (es su destino)  & Llegar a destino\\\cline{2-4}
     1 & Seguir ruta & Celda disponible  & Moverse\\\cline{2-4}
     0 & Seguir ruta & En un cruce con semáforo en estado rojo & Esperar \\\cline{2-4}
    \end{tabular}
    \caption{Arquitectura de subsunción para los coches}
    \label{tab:arch}
\end{table}

\section{Características del Ambiente}
El espacio en el que viven los agentes es discreto y finito, el cual se representa con un 
tablero de $M\times N$ casillas,y este se crea a partir de un documento de texto, en el 
que se determinan los sentidos de las calles, las posiciones de los semáforos y los destinos
a partir de una simbología previamente determinada. 

El ambiente es relativamente inaccesible, ya que aunque los agentes son capaces de obtener
rutas eficientes (en términos de distancia) y por ende tienen un cierto conocimiento del mapa,
no son capaces de predecir el comportamiento ni la ubicación de otros agentes (a excepción de sus)
vecinos más próximos. 

El ambiente es un tanto estático; ya que aunque se introducen, eliminan y se mueven agentes vehiculares,
los semáforos y los destinos no cambian de posición. Asimismo, el ambiente es no determinista, ya que
el orden en el que los agentes toman las decisiones cambia entre simulaciones (debido al uso del {\tt{RandomScheduler}} de 
mesa) al igual que por ciertas variables aleatorias\footnote{En realidad son pseudo-aleatorias dado que son números generados computacionalmente y por ende son replicables con cierto conocimiento de la semilla y la secuencia.} que se utilizan para definir la tolernacia de los agentes.

\section{Controlador de Agentes}
Para la parte gráfica de la simulación se utilizó Unity como herramienta, en donde se hizo la visualización de todos los agentes como vehículos puntualmente utilizando wavefronts (obj) como los modelos utilizados. 
El control de los agentes se hizo en su totalidad con la librería Mesa de Python, conectando la funcionalidad de la librería con Unity a través de un servidor de Flask. La forma en la que esto funciona es que se le brinda a 
Unity la capacidad de hacer peticiones al servidor que con el script AgentController, que es el responsable de recibir la posición de todos los agentes como existan en Mesa, para que con los scripts de movimiento, desplace los agentes 
con transformaciones de matrices, otra función del script es hacer que se instancíen los objetos en Unity para poder controlarlos, así como destruirlos cuando lleguen a su destino.

\section{Resultados y conclusiones}

Originalmente se consideró una arquitecutra de subsunción en la que los agentes recalculaban una ruta cada vez que un vehículo obstruía la celda
a la que se querían mover. Sin embargo, debido a que en el mapa había una glorieta, algunos agentes se notaban impacientes por desplazarse y por lo tanto se metían al
carril interno. Como los agentes propuestos no cuentan con direccionales, a veces les resultaba imposible llegar a su salida ya que nadie cedía el paso en favor de su proactividad
y con una afluencia alta de agentes terminaban en un \textit{gridlock}. Después de un bloqueo de la rotonda, la simulación sobrevivía unas cuantas decenas de episodios más antes de 
que todas las calles quedaran congestionadas y que consecuentemente se llegara a la condición de cierre.

El bloqueo entre los vehículos dio pauta al principal descubrimiento de esta investigación, la tolerancia es todo lo que se necesita. Pues resultó evidente con la rotonda que cambiarse de carril no
siempre es la mejor decisión y si no se introducen algoritmos de aprendizaje que le permitan a los agentes determinar cuándo es conveniente cambiar de carril, resulta eficiente
que esperen unos episodios antes de tomar la decisión. Además, para evitar que muchos coches se cambien de carril de forma simultánea, se generaron sus límites de tolerancia de forma
pseudo-aleatoria y en caso de que este límite de paciencia se rebase, se tiene un segundo límite más agresivo con el que un vehículo preferirá moverse a cualquier espacio disponible: no solo cambiando su ruta 
por completo; sino también sacrificando su proactividad en favor del rendimiento de la simulación en su totalidad. Esta decisión, aunque parece ser irracional podría en realidad confundirse por un comportamiento
inteligente, como si los agentes pudieran entender la existencia de un equilibrio de Nash. 

Una vez realizados estos cambios y trás la observación de su rendimiento desde el visualizador de {\tt{mesa}}, se decidió correr la simulación múltiples veces
como proceso en segundo plano (agregando agentes cada 2 episodios) y escribir los resultados a un archivo para encontrar patrones en su rendimiento (ver \autoref{fig:runs}). 
La principal observación que se obtuvo de estos resultados fue que inicialmente la llegada de los agentes a sus destinos es rápida y casi lineal, pero eventualmente se 
estanca mostrando un comportamiento asintótico y dando fin a la simulación.

\begin{figure}[ht]
  \begin{subfigure}[b]{0.32\textwidth}
  {\begin{tikzpicture}[scale=0.55]
  \begin{axis}[
      grid style={line width=.1pt, draw=gray!10},
      major grid style={line width=.2pt,draw=gray!50},
      xlabel={Episodio},
      ylabel={Número de llegadas totales},
      ymin={0},
      ymax={2000},
      xmin={0},
      xmax={1300}
  ]
  \addplot[red] table[x=step, y=arrivals]{./data/run1.txt};
  \end{axis}
  \end{tikzpicture}}
  \caption{Primera Simulación}
  \end{subfigure}
  \hfill
  \begin{subfigure}[b]{0.32\textwidth}
    {\begin{tikzpicture}[scale=0.55]
    \begin{axis}[
        grid style={line width=.1pt, draw=gray!10},
        major grid style={line width=.2pt,draw=gray!50},
        xlabel={Episodio},
        ylabel={Número de llegadas totales},
        ymin={0},
        ymax={2000},
        xmin={0},
        xmax={1300}
    ]
    \addplot[green] table[x=step, y=arrivals]{./data/run2.txt};
    \end{axis}
    \end{tikzpicture}}
    \caption{Segunda Simulación}
  \end{subfigure}
  \hfill
  \begin{subfigure}[b]{0.32\textwidth}
    {\begin{tikzpicture}[scale=0.55]
    \begin{axis}[
        grid style={line width=.1pt, draw=gray!10},
        major grid style={line width=.2pt,draw=gray!50},
        xlabel={Episodio},
        ylabel={Número de llegadas totales},
        ymin={0},
        ymax={2000},
        xmin={0},
        xmax={1300}
    ]
    \addplot[blue] table[x=step, y=arrivals]{./data/run3.txt};
    \end{axis}
    \end{tikzpicture}}
    \caption{Tercera Simulación}
  \end{subfigure}
  \caption{Resultados agregando agentes cada 2 episodios y usando semáforos con tiempos de 10s entre estados}
  \label{fig:runs}
\end{figure}

\newpage

\bibliographystyle{apacite}
\bibliography{refs}
\end{document}
