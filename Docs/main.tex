\documentclass[twoside,11pt]{article}
\usepackage[spanish]{babel}
\usepackage{xcolor}
\usepackage{style}
\usepackage{tabularx}
\usepackage{csquotes}
\usepackage{subcaption}
\usepackage{wrapfig}
\usepackage{graphicx}
\usepackage{float}
\usepackage{apacite}
\usepackage{bm}

\newcommand{\myname}{Joaquín Badillo Granillo}

\jmlrheading{Noviembre 29}{2023}{Movilidad Urbana}{TC2008B}
\ShortHeadings{Movilidad Urbana}{}

\begin{document}
\renewcommand{\tablename}{Tabla}

\title{Simulación Multiagente de Movilidad Urbana}

\author{\name \myname \email \href{mailto:a01026364@tec.mx}{a01026364@tec.mx} \\
       \addr Escuela de Ingeniería y Ciencias\\
       Tecnológico de Monterrey\\
       Campus Santa Fe
       \AND
       \name Pablo Bolio Pradilla \email \href{mailto:a01782428@tec.mx}{a01782428@tec.mx} \\
       \addr Escuela de Ingeniería y Ciencias\\
       Tecnológico de Monterrey\\
       Campus Santa Fe
       }

\maketitle

\begin{abstract}
    
\end{abstract}

\begin{keywords}
  Simulación, Multiagentes, Agentes Reactivos, Movilidad
\end{keywords}

\section{Introducción}
El rápido crecimiento urbano ha dado como resultado altas densidades poblacionales.
En zonas centralizadas como la Ciudad de México, aunque la cercanía de los servicios
es muy alta, la cantidad de vehículos que se trasladan por las calles ha hecho que
el tiempo para llegar a dichos servicios sea muy largo en comparación con su distancia.

\subsection{Propuesta de Solución}
En este proyecto se consideró el problema de la movilidad urbana a partir de una simulación
multiagente utilizando la librería de mesa \cite[]{python-mesa-2020}. En esta simulación,
los agentes toman decisiones con un cierto grado de autonomía a partir de un conjunto de reglas
finitas y sencillas, como el respetar las señales de tránsito (en particular los semáforos) y 
la navegación a partir de las rutas más cortas para alcanzar un destino.

Es entonces de nuestro interés observar si a partir de estas sencillas reglas y limitando el entendimiento 
que tienen los agentes del ambiente a pequeñas vecindades, pueden emerger comportamientos egoístas que resulten
en congestionamientos. Asímismo, se planteó una simulación en la que es posible modificar la afluencia de agentes,
de tal forma que se puedan observar los límites que tiene el diseño de una ciudad tras un incremento en la densidad
poblacional.

\section{Diseño de Agentes}

Para que los agentes puedan interactuar con su entorno, se representaron calles, edificios, destinos, semáforos y vehículos
como agentes.

Tanto los semáforos como los vehículos son capaces de cambiar su estado, pero únicamente los segundos tienen
un objetivo: que es llegar su destino. Por esta razón, el diseño que es de nuestro interés es el de los vehículos, el cual
se describe a detalle a continuación.

\subsection{Capacidad Efectora (Actuadores)}

Las acciones que puede realizar un vehículo están limitadas a:

\begin{enumerate}
    \item Esperar.
    
    \item Moverse.

    \item Calcular una ruta.

    \item Llegar a destino.
\end{enumerate}

Esperar, moverse y llegar a su destino, son acciones sencillas y fáciles
de implementar; mientras que el calculo de rutas permite una mayor libertad
creativa. En esta simulación, los agentes calculan rutas a partir del algoritmo 
de A*, usando generalmente el cuadrado de la norma $\ell^{2}$ como función de costos y 
la norma $\ell^{1}$ como heurística. En caso de recalcular una ruta, un agente puede
además usar la información de sus vecinos más próximos para incrementar el costo
entre celdas.

\subsection{Percepción}
Los vehículos son únicamente capaces de observar una vecindad en un radio de $\sqrt{2}$
unidades (usando la norma $\ell^{2}$ como métrica). No obstante cuentan con un conocimiento
previo de las calles que les permite calcular rutas eficientes (asumiendo que no hay otros
vehículos).

\subsection{Proactividad}

\subsection{Métricas de Desempeño}
Si fuera de nuestro interés determinar el desempeño individual de un agente, una métrica relevante sería 
la cantidad de episodios que le tomó llegar a su destino, aunque debido a que la distancia que tienen que 
recorrer varía es aún mejor evaluarlos individualmente como
$$\frac{\mathrm{Episodios \ Transcurridos}}{\langle\mathrm{Origen}, \mathrm{Destino}\rangle_{\ell^{1}}},$$
donde $\langle \bm{x}, \bm{y}\rangle_{\ell^{1}}$ representa la distancia Manhattan entre 2
puntos $\bm{x}$, $\bm{y}$.

No obstante, como es de nuestro interés determinar el rendimiento de los agentes a un nivel global, es más
razonable evaluar el rendimiento que tienen de forma colaborativa y como responden a cambios en el flujo de
entrada de agentes al sistema. Por esta razón se decidió que la simulación se detuviera cuando no se pudiera
ingresar más agentes en ninguna de las esquinas del modelo (que es donde aparecen los nuevos agentes) y se 
midió el rendimiento para un flujo de entrada particular como el número de episodios que podían transcurrir sin
llegar a la condición de cierre.

$$R(\phi_{\mathrm{e}}) = \mathrm{Episodios \ Totales}$$.

El flujo de entrada en la simulación, se representó como un parámetro de línea de comandos para el servidor, que 
determina el número de episodios que debe transcurrir antes de volver a agregar agentes.

\section{Arquitectura de Subsunción}\label{sec:arch}
\begin{table}[ht]
    \centering
    \begin{tabular}{c| >{\raggedright}p{0.3\textwidth} >{\raggedright}p{0.3\textwidth} >{\raggedright\arraybackslash}p{0.3\textwidth}|}
    \cline{2-4}
      & \textbf{Evento} & \textbf{Condiciones} & \textbf{Acción} \\\cline{2-4}
     4 & Nada & Nada & Calcular ruta \\\cline{2-4}
     3 & Seguir ruta & Celda disponible (es su destino)  & Llegar a destino\\\cline{2-4}
     2 & Seguir ruta & Celda disponible & Moverse siguiendo ruta\\\cline{2-4}
     1 & Seguir ruta & Celda ocupada por otro coche & Calcular ruta (actualizando costos para las celdas vecinas)\\\cline{2-4}
     0 & Seguir ruta & En un cruce con semáforo en estado rojo & Esperar \\\cline{2-4}
    \end{tabular}
    \caption{Arquitectura de subsunción para los coches}
    \label{tab:arch}
\end{table}

\section{Características del Ambiente}
\section{Conclusiones}

\newpage

\bibliographystyle{apacite}
\bibliography{refs}
\end{document}